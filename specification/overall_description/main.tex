\section{Allgemeine Beschreibung}


\subsection{Produktperspektive}


Beim entwickelten Produkt handelt es sich um eine komplette Neuentwicklung. Architekturell handelt es sich um eine Client-Server Anwendung mit eine Browser Applikation welche mit einem Server kommuniziert, der wiederum alle benötigten Daten aufbereitet und persistiert.

\subsection{Produktfunktionen}

\subsubsection{Eintragen von Verbräuchen}

Der primäre Anwendungsfall der Applikation ist das Eintragen und Verwalten von verschiedenartigen Verbrauchswerten. Hierzu zählen unter anderem aber nicht ausschließlich Gas, Strom und Wasser.

\paragraph{Manuelles Eintragen}

Verbrauchswerte können über das Web-Interface eingetragen werden.

\paragraph{Verbraucher}
Verbraucher, bei denen der Verbrauch konstant ist können ebenfalls über das Webinterface eingetragen werden.

\paragraph{Automatisiertes Eintragen}
\label{autom_eintragen}

Verbrauchswerte können zudem über Externe Geräte von Drittanbietern direkt am Zähler ausgelesen werden. Die Werte werden dann vom Gerät über eine eine von der Applikation bereitzustellende API übertragen.

\subsubsection{Statistiken}

Alle gesammelten Daten können im Web-Interface auf verschiedene Arten visualisiert und mit anderen Daten verglichen werden. Der Vergleich soll sowohl anhand des eigentlichen Verbrauchs (also in beispielsweise kWh) aber auch umgerechnet in Kosten in der jeweiligen Landeswährung oder Umweltbelastung in beispielsweise Tonnen $CO_2$ möglich sein.

\paragraph{Vergleich mit eigener Historie}

Zunächst kann der Benutzer seinen aktuellen Verbrauch  mit (von Ihm ausgewählten) Vergleichszeitreihen aus seiner eigenen Verbrauchshistorie vergleichen.


\paragraph{Vergleich mit anderen Nutzern der Plattform}
Darüber hinaus kann der Nutzer seinen Momentanverbrauch auch mit gemittelten Werten aus Vergleichszeitreihen die aus diversen Nutzergruppen mit bestimmten Vergleichstags generiert werden. Die Vergleichstags sind variabel und müssen durch den Systemadministrator einstellbar sein. Vorstellbare Vergleichstags sind beispielsweise eine Aufteilung der Nutzer nach Wohnsitz oder Größe der Immobilie.


\paragraph{Vergleich mit plattformunabhängigen Metriken}
\label{vgl_plattformunabhängig}

Die letzte mögliche Vergleichsmetrik besteht in Standard Vergleichswerten welche nicht von Nutzern dieser Applikation, sondern von externen Forschungsinstituten bereitgestellt werden.
Diese Daten werden aus dem Format des externen Instituts zur internen Verwendung in eine Vergleichszeitreihe übertragen, um unabhängig von der Auslieferungsart der Daten des Instituts zu sein.

\subsection{Benutzermerkmale}\label{sec:desc_user}
% menschliche Akteure + Eigenschaften (zB Rechte)
% zB Admin: kompliziertere Oberflächen, kann zum benutzen geschult werden. Wie Support

Der Nutzerkreis wird auf Privatpersonen eingeschränkt.
Zur Nutzung der App muss eines von drei Abomodellen (FREE, STANDARD, PROFESSIONAL) ausgewählt werden.
Die Aufteilung folgt einem \gquote{Freemium} Schema, sprich es gibt eine kostenlose Klasse und zwei Bezahlmodelle. Die Abgrenzung zwischen den Klassen bleibt zu entscheiden. Vorstellbar ist hier eine Begrenzung der Anzahl von Immobilien im Free-Modell, oder etwa das Features wie das automatisierte Eintragen von Verbrauchswerten (siehe \ref{autom_eintragen}) nur für bezahlende Nutzer zur Verfügung stehen.

Neben dem Benutzer soll es auch weitere Benutzertypen geben - dem Support- und Admin-Benutzer. Diese können zur Bedienung des Produkts geschult werden.

\subsection{Entwicklungsumgebung}
% Wo gehostet? Wir können "völlig frei" schreiben.

\subsubsection{Server}

Für die Laufzeitumgebung des Servers gibt es keine Einschränkungen.

\subsubsection{Client}
\label{subsec:OEclient}
Die Client-Anwendung wird als Browser-Anwendung zur Verfügung gestellt. Sie muss alle gängigen Browser (Google-Chrome und Chromium basierte Browser, Firefox, Safari) auf allen gängigen Endgeräten (Smartphone, Tablet, PC) mit den jeweiligen Betriebssystemen (Android / iOS, Windows / MacOS / Linux) unterstützen.

\subsection{Design- und Implementierungsbeschränkungen}

Wie bereits in \ref{subsec:OEclient} beschrieben ist die Client-Anwendung als Browser-Applikation zu implementieren. Darüber hinaus muss die Nutzeranwendung einfach und intuitiv bedienbar sein da kein Fachpublikum angesprochen wird.

\subsection{Benutzerhandbuch/Nutzerdokumentationen}

Da die Client Applikation intuitiv bedienbar sein muss, sollte eine Dokumentation für den Großteil der Nutzer nicht notwendig sein. Da jedoch auch technisch weniger versierte Nutzer zur Zielgruppe gehören und um spezielle Anwendungsfälle abzudecken wird eine Dokumentation in Form eines Online-Wiki direkt auf der Website der Anwendung bereitgestellt. Darüber hinaus können Fragen auch vom Kundensupport beantwortet werden.

\subsection{Annahmen und Abhängigkeiten}
% Annahmen über die Welt
% zB "Wir gehen davon aus, dass alle Geräte über HTTP kommunizieren"
% zB "Wir gehen davon aus dass alle Messwerte in SI Einheiten sind"


\subsubsection{Statistiken}
Plattformunabhängig Statistiken (siehe \ref{vgl_plattformunabhängig}) werden von Drittanbietern bereitgestellt. Das Format, die Aktualität der Daten und die Kontinuität der Bereitstellung kann nicht sichergestellt werden.


\subsubsection{Externes Gerät}
Die Implementierung von kompatiblen externen Geräten (siehe \ref{autom_eintragen}) ist abhängig von Drittanbietern.

\subsubsection{Zahlungsanbieter}

Um Nutzern das Bezahlen so komfortabel wie möglich zu machen greifen wir auf externe Zahlungsdienstleister wie beispielsweise PayPal, Klarna, Visa usw. zurück. Die Schnittstellen dieser Anbieter können sich ändern, sowie deren Geschäftsbedingungen, Kosten und Anforderungen.
