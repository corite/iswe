\section{Anforderungen an externe Schnittstellen}
\subsection{Benutzeroberflächen}
% 3.1 und 3.4 überschneiden sich. 
% 3.4 eher über Rechnergrenzen und 3.1 nur lokal
% haben wir aber schon richtig gemacht
\subsubsection{Benutzeroberfläche}
Die Software bietet ein User-Interface an, welches Zugriff auf alle Standard Funktionen, wie Verbrauch eintragen,
Verbrauch vergleichen oder Statistiken anzeigen, bereitstellt.
\subsubsection{Supportoberfläche}
Über ein Support-Interface soll erweiterter Zugriff auf User-Accounts nach deren Freigabe möglich sein.
\subsubsection{Adminoberfläche}
Es wird eine Admin-Oberfläche bereitgestellt, welche vollkommene Kontrolle über alle Funktionalitäten hat.
Insbesondere soll es möglich sein, Vergleichszeitreihen anzulegen und die Funktionalitäten,
die den Abo-Modellen zugeordnet sind, anzupassen.
\subsection{Hardware-Schnittstellen}
Die Software bedient sich keiner Hardware-Interfaces.
\subsection{Software-Schnittstellen}
\subsubsection{Stadtwerke}
Die Software bezieht, sofern vorhanden, Daten aus den örtlichen Stadtwerken um Statistiken bereitzustellen,
mit denen der eigene Verbrauch verglichen werden kann.
\subsubsection{Zahlungsdienstleister}
Ein Austausch mit den Zahlungsdienstleistern ist erforderlich, um Zahlungen für kostenpflichtige Abomodelle
gewährleisten zu können.
\subsection{Kommunikationsschnittstellen}
Die Kommunikation der einzelnen Interfaces erfolgt über HTTPS.