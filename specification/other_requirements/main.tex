\section{Andere Anforderungen}
\subsection{Glossar}
\begin{itemize}
\item Benutzer: Als Benutzer beschreiben wir einen Konsumenten des Softwareprodukts.
\item Supporter: Als Supporter beschreiben wir eine angestellte Hilfsperson, mit eingeschränkten Berechtigungen, die dem Benutzer bei Problemen helfen kann.
\item Admin: Der Admin ist eine angestellte Person, die Vollzugriff auf das gesamte System hat.
\item Externes Gerät: Das Externe Gerät, ist in der Lage automatisch Strom vom Stromzähler eines Benutzers einzulesen.
\item Immobilie: Immobilien sind in unserer Beschreibung Immobilien in denen der Benutzer seine Verbräuche erfassen möchte.
\item Verbrauchstyp: Der Verbrauchstyp ist die Art des Verbrauchs, zum Beispiel: Strom.
\item Einheit: Die Einheit beschreibt eine tatsächliche Einheit eines Verbrauchstyps, zum Beispiel: Kilowattstunde
\item Einseinheit: Die Einseinheit ist eine Einheit mit dem Umrechnungsfaktor 1. Diese muss für jeden Verbrauchstyp definiert sein.
\item Verbrauch: Der Verbrauch modelliert den tatsächlichen Wert des Verbrauchs in der Einseinheit über einen gewissen Zeitraum.
\item Zeitreihe: Eine Zeitreihe modelliert einen Verbrauch mit einem Verbrauchstyp.
\item Vergleichszeitreihe: Als Vergleichszeitreihen beschreiben wir Zeitreihen, die wir aus Nutzerdaten generieren oder aus externen Quellen erstellen. Diese dienen lediglich zum Vergleichen des Verbrauchs eines Benutzers.
\item Verbraucher: Ein Verbraucher ist ein Gerät einer Immobilie welches einen konstanten Verbrauch hat und somit automatisch täglich in eine Verbrauchszeitreihe eingetragen werden kann.
\end{itemize}
\subsection{Modelle}