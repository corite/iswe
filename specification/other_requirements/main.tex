\section{Andere Anforderungen}
Es sind keine anderen Anforderungen bekannt.
\newpage
\section{Anhang}
\subsection*{Anhang A: Glossar}
\addcontentsline{toc}{subsection}{Anhang A: Glossar}
\label{glossar}
\begin{itemize}
	\item Nutzer/Benutzer: Als Benutzer beschreiben wir einen Konsumenten des Softwareprodukts.
	\item Support/Supporter: Als Supporter beschreiben wir eine angestellte Hilfsperson, mit eingeschränkten Berechtigungen, die dem Benutzer bei Problemen helfen kann.
	\item Admin: Der Admin ist eine angestellte Person, die Vollzugriff auf das gesamte System hat.
	\item Nutzergruppe: Eine Nutzergruppe ist entweder die Menge aller Benutzer, Supporter oder Admins.
	\item Support Ticket: Ein Support Ticket ist ein Objekt welches die Kommunikation zwischen Supporter und Benutzer zu einem bestimmten Thema bündelt.
	\item Externes Gerät: Das Externe Gerät, ist in der Lage automatisch Strom vom Stromzähler eines Benutzers einzulesen.
	\item Immobilie: Immobilien sind in unserer Beschreibung Immobilien in denen der Benutzer seine Verbräuche erfassen möchte.
	\item Verbrauchstyp: Der Verbrauchstyp ist die Art des Verbrauchs, zum Beispiel: Strom.
	\item Einheit: Die Einheit beschreibt eine tatsächliche Einheit eines Verbrauchstyps, zum Beispiel: Kilowattstunde
	\item Einseinheit: Die Einseinheit ist eine Einheit mit dem Umrechnungsfaktor 1. Diese muss für jeden Verbrauchstyp definiert sein.
	\item Verbrauch: Der Verbrauch modelliert den tatsächlichen Wert des Verbrauchs in der Einseinheit über einen gewissen Zeitraum.
	\item Zeitreihe: Eine Zeitreihe modelliert einen Verbrauch mit einem Verbrauchstyp.
	\item Vergleichszeitreihe: Als Vergleichszeitreihen beschreiben wir Zeitreihen, die wir aus Nutzerdaten generieren oder aus externen Quellen erstellen. Diese dienen lediglich zum Vergleichen des Verbrauchs eines Benutzers.
	\item Verbraucher: Ein Verbraucher ist ein Gerät einer Immobilie welches einen konstanten Verbrauch hat und somit automatisch täglich in eine Verbrauchszeitreihe eingetragen werden kann.
\end{itemize}
\subsection*{Anhang B: Modelle}\label{sec:app_modelle}
\addcontentsline{toc}{subsection}{Anhang B: Modelle}
\subsubsection*{Use-Case Diagramm}
\includegraphics[scale=0.4]{use_case.png}

\subsubsection*{Klassendiagramm}
\includegraphics[scale=0.3]{klassendiagramm.png}

\subsection*{Anhang C: Offene Punkte}
\addcontentsline{toc}{subsection}{Anhang C: Offene Punkte}

\begin{itemize}
	\item Die Zuteilung der Features zu den Abomodellen.
	\item Die konkrete Wahl der unterstützten Zahlungsanbietern.
	\item Alle funktionalen Anforderungen die mit \gquote{Noch zu diskutieren} beschrieben sind.
	\item Die konkreten Anforderungen an die Schnittstelle für externe Geräte.
\end{itemize}

