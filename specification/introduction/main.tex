\section{Einleitung}
\subsection{Zweck}
% Zweck des Dokuments. Was ist das Dokument?
Dieses Dokument beschreibt das Softwareprojekt \gquote{Verbrauchstransparenz} in seiner initialen Version.
Das Dokument hat den Anspruch das Projekt in seiner Gesamtheit inklusive aller Subsysteme abzubilden.

\subsection{Richtlinien des Dokuments}
% Wie ist das Dokument formatiert? zB Fachbegriffe groß geschrieben
Das Dokument als Ganzes folgt keinen speziellen Konventionen oder Richtlinien. %TODO stimmt das noch?
Falls jedoch in bestimmten Kapiteln Konventionen oder Richtlinien Anwendung finden, wird dies dort hervorgehoben. 

\subsection{Zielgruppe und Leseempfehlung}
% Wer ist der intendierte Leser. zB Kunde / Auftraggeber, Entwickler
\subsubsection{Entwickler}

Entwickler können dieses Dokument nutzen um Informationen über die Anforderungen an die von Ihnen hergestellte Software zu bekommen. 

Grundlegend sind für sie alle Kapitel relevant (je nach Abstraktionsebene), es bietet sich an diese in Reihenfolge zu lesen um von generellen Aspekten zu spezifischen Details überzugehen. Für die Implementierung ist Kapitel 4 von besonderem Interesse, da es die einzelnen Funktionalitäten detailliert beschreibt.

\subsubsection{Kunde}

Der Kunde kann aus diesem Dokument den aktuellen Stand der Projektplanung verfolgen, außerdem dient es Ihm um zu überprüfen ob seine Anforderungen korrekt verstanden wurden und die Prioritäten des Entwicklungsteams mit seinen Eigenen übereinstimmen.

Auch für den Kunden sind grundsätzlich alle Sektionen relevant, jedoch sollte er besonderes Augenmerk auf die Kapitel mit hohem Abstraktionsgrad legen, wie zum Beispiel die Allgemeine Beschreibung und die System- und Qualitätsanforderungen.

\subsubsection{Tester}

Tester können dieses Dokument nutzen um die konkreten Anforderungen an das Produkt und somit auch das zu testende Verhalten besser zu verstehen. 

Für sie ist Kapitel 4 von besonderer Bedeutung, um einen Gesamtüberblick zu bekommen kann es aber auch sinnvoll sein die anderen Teile des Dokuments zu lesen.%TODO sollen hier auch noch qualitätsanforderungen hin?

\subsection{Umfang}
% Was ist das Produkt

Die Software gibt Benutzern die Möglichkeiten ihre Verbräuche, primär Strom, Gas und Wasser, aber auch individuelle Verbrauchswerte im Blick zu haben und diese mit Anderen zu vergleichen.

Dabei stehen dem Benutzer 3 verschiedene Abomodelle (FREE, STANDARD, PROFESSIONAL), unter denen er wählen kann, zur Verfügung. Diese bestimmen die Möglichkeiten der Datenerhebung, den Grad der Auswertung dieser Daten und deren Speichermöglichkeiten.

Für die Nutzer entsteht ein Mehrwert zum einen durch das Erfüllen seines Informationsinteresses am eigenen Verbrauch und zum anderen durch das Erkennen von Einsparpotentialen im Haushalt.

Darüber hinaus soll die Nutzung des Produkts auch zum Energiesparen im Sinne des Umweltschutzes beitragen.

\subsection{Verweise auf sonstige Ressourcen oder Quellen}

Dieses SRS-Dokument erhebt den Anspruch der Vollständigkeit. Aus diesem Grund sind keine weiteren Quellen oder sonstige zusätzliche Ressourcen von Nöten.
