\section{Introduction}
\subsection{Purpose}
% Zweck des Dokuments. Was ist das Dokument?
Das Produkt soll Nutzern die Möglichkeit geben, Verbrauch zu dokumentieren, insbesondere Strom, Gas und Wasser. Der Nutzer hat mit diesem Produkt die Möglichkeit seinen Verbrauch zu vergleichen.
\subsection{Document Conventions}
% Wie ist das Dokument formatiert? zB Fachbegriffe groß geschrieben
I dont know, haben wir bisher nicht wirklich, könnte mir aber vorstellen, dass später so Systemanforderungen hier schon hin könnten, oder wovon wir ausgehen, was genutzt wird.
\subsection{Intended Audience and Reading Suggestions}
% Wer ist der intendierte Leser. zB Kunde / Auftraggeber, Entwickler
Entwickler: Überblick über Projekt\\
Kunde: Überblick über Projektfortschritt und Features\\
Benutzer: Wissen darüber, was das Produkt kann\\
Tester:  Wissen darüber, was das Produkt können soll\\
Dieses SRS enthält Informationen und Programmdetails.
Ist so und so organisiert und sollte so und so vielleicht gelesen werden.
\subsection{Product Scope}
% Was ist das Produkt
Die Software gibt Benutzern die Möglichkeiten ihre Verbräuche, primär Strom, Gas und Wasser, aber auch individuelle Verbräuche im Blick zu haben und diese mit anderen zu vergleichen.\\
Dabei stehen dem Benutzer 3 verschiedene Abomodelle, unter denen er wählen kann, zur Verfügung, welche die Möglichkeiten der Datenerhebung, den Grad der Auswertung dieser und die Datenspeicherung einschränken.\\
Für den Kunden werden die Ziele des Geldverdienens und das er Benutzern hilft Ihre Verbräuche zu tracken und gegebenenfalls Geld zu sparen erfüllt.
\subsection{References}
Haben wir noch nicht.