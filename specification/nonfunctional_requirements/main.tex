\section{Nicht-Funktionale Anforderungen}

\subsection{Leistungsanforderungen}
\textit{If there are performance requirements for the product under various 
circumstances, state them here and explain their rationale, 
to help the developers understand the intent and make suitable design choices. 
Specify the timing relationships for real time systems. 
Make such requirements as specific as possible. 
You may need to state performance requirements 
for individual functional requirements or features.}

\textbf{wahrscheinlich so lightweight wie möglich, 
weil potentiell jeder Mensch die Webseite erreichen können muss}


\subsection{Sicherheitsanforderungen}
\textit{Specify those requirements that are concerned 
with possible loss, damage, or harm that could result from the use of the product. 
Define any safeguards or actions that must be taken, as well as actions that must be prevented. 
Refer to any external policies or regulations that state safety issues that affect the product’s design or use. 
Define any safety certifications that must be satisfied.}


Bei einem Datenbankbreach sollten die sensiblen Daten unserer Nutzer nicht 
gefährdet werden.
Deswegen dürfen die Kontodaten, Passwörter und Adressen nicht zusammen gespeichert werden.
Weiterhin werden die Account-, Bezahl- und Verbrauchsdaten jeweils in verschiedenen Datenbanken ((gespeichert)).
%Risikobewertung vs Kosten??

Im Fall eines Ausfalles des ((Systems)) sollte nach spätestens zwei Stunden
eine Reaktion von ((uns)) kommen, 
Bemühungen das System ((zum laufen zu bekommen)).
Sechs Stunden Ausfall sind zu viel, das muss vermieden werden.

Daten müssen in mindestens zwei Verschiedenen ((Orten)) 
abgespeichert werden, 
damit bei einem Verlust von Daten % wie können wir das verhindern?
möglichst wenig verloren geht.
Es sollen maximal alle fünf Minuten die Daten gespeichert werden, 
um den Datenverlust gering zu halten. %sollen wir das begründen?




\subsection{Datenschutzanforderungen}
Die Datenschutzanforderungen sollen der Datenschutzgrundverordnung entsprechen. Bei einem Datenbankbreach sollen die sensiblen Daten der Nutzer nicht gefährdet werden. Deswegen dürfen die Kontodaten, Passwörter und Adressen nicht zusammen gespeichert werden. Weiterhin werden die Account-, Bezahl- und Verbrauchsdaten jeweils in verschiedenen Datenbanken gespeichert. Passwörter werden zusätzlich nur als Hashwerte abgespeichert, welche dann nur noch abgeglichen werden müssen.\\
Es sollen möglichst wenig persönliche Daten erhoben werden müssen und viele sollen optional sein:

\begin{enumerate}
    \item Ein Nutzername und das zugehörige Passwort müssen abgespeichert werden um den Nutzer eindeutig zuzuordnen
    \item Der Nutzer kann seine Kreditkartendaten abspeichern, muss er aber nicht
    \item Der Nutzer kann seine Adresse angeben, muss er aber nicht
    \item Zu Immobilien können Adressen gespeichert werden sind aber optional
    \item Eingetragene Daten zu einer Immobilien werden pro Benutzer in einer Datenbank gespeichert
\end{enumerate}
Die Verbräuche eines Nutzers werden zusätzlich anonym in einer Datenbank für Vergleiche festgehalten. Alle Daten von Gas-, Strom- und Wasserverbräuche werden ohne Nutzerbezug abgespeichert. Diese Daten dienen Verbrauchsstatistiken mit denen sich jeder Nutzer vergleichen kann. Es wird Wert darauf gelegt, dass sich diese Daten nicht zurückverfolgen lassen, aufgrund der Regionalität. Sollten zu wenig Daten für einen regionalen Vergleich vorliegen, wird dieser nicht angeboten.

\subsection{Software-Qualitätsattribute}
\textit{Specify any additional quality characteristics for the product that will be important to either the customers or
the developers.
Some to consider are: adaptability, availability, correctness, flexibility, 
interoperability, maintainability, portability, reliability, reusability, 
robustness, testability, and usability. Write these to be specific, quantitative, 
and verifiable when possible.
At the least, clarify the relative preferences for various attributes, 
such as ease of use over ease of learning.}
% Auf 4-5 wichtige Eigenschaften einigen, welche einander nicht ausschließen
% Zur Auswahl: 
% Anpassungsfähigkeit, Verfügbarkeit, Richtigkeit, Flexibilität,
% Kombatibilität, Wartungsfähigkeit, Portabel, Zuverlässig,
% Wiederverwendbarkeit, Robust, Testbar, Nützlich/ Benutzerfreundlich? 

% Spezifisch, messbar, bestätigbar?

\begin{itemize}
    \item \textbf{Verfügbarkeit: }
    Die abzurufenden Daten sollten zu jederzeit verfügbar sein,
    damit der Benutzer immer die Möglichkeit hat, 
    seine Verbäuche einzusehen und vergleichen zu können.
    
    \textit{(ELI: immer verfügbar, max 2h Ausfall bis Reaktion, 
    max 6h bevor Stress kommt(müssen wir uns für sowas einen Plan ausdenken??)\\,
    es dürfen nur ca 5 min an Daten verloren gehen\\
    messbar: Trainingsszenario? bei dem geschaut wird, 
    wie viele Daten verloren gehen oder sowas? )}
    
    \item \textbf{Richtigkeit: }
    Es sollten die korrekten, 
    vom Nutzer verbrauchten Daten einsehbar und 
    aktuell sein und angezeigt werden.

\textit{(ELI: kann man das spezifizieren?? unrealistische Angaben überprüfen?
\\TestNutzer anlegen, austesten, 
    wie sich die Daten verhalten und errechnen, 
    was eigentlich rauskommen sollte)}
    \item \textbf{Wartungsfähigkeit: }
    Die Administratoren und der Support sollten korrekte Verbrauchsdaten 
    pflegen und die Anwendung jederzeit an ein Abo-Modell anpassbar sein.

    \textit{(ELI: heißt das auch, dass der Code übersichtlich, verständlich, 
    nachvollziehbar ist? Kann man sowas messen?
    Probleme/Bugs müssen schnell erkannt und gefixt werden?? )}

    \item \textbf{Benutzerfreundlichkeit: }
    Die angezeigten Daten sollten möglichst viele Kundenbedürfnisse 
    befriedigen, übersichtlich und leicht verständlich sein.

    \textit{(ELI: könnte verschiedene Versionen des Designs an 
    Leute verteilen, die abstimmen, bewerten, kritisieren?)}
\end{itemize}

\subsection{Geschäftsregeln}
\textit{List any operating principles about the product, 
such as which individuals or roles can perform which functions under specific circumstances. 
These are not functional requirements in themselves, 
but they may imply certain functional requirements to enforce the rules}

    Es gibt 3 Rollen mit jeweils unterschiedlichen Berechtigungen im System: Benutzer, Support und Admin. Die detaillierte Aufteilung der Befugnisse ist in \ref{system_features} festgelegt.
