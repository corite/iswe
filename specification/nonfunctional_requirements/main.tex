\section{Nicht-Funktionale Anforderungen}

\subsection{Leistungsanforderungen}
\textit{If there are performance requirements for the product under various 
circumstances, state them here and explain their rationale, 
to help the developers understand the intent and make suitable design choices. 
Specify the timing relationships for real time systems. 
Make such requirements as specific as possible. 
You may need to state performance requirements 
for individual functional requirements or features.}


\subsection{Sicherheitsanforderungen}
\textit{Specify those requirements that are concerned 
with possible loss, damage, or harm that could result from the use of the product. 
Define any safeguards or actions that must be taken, as well as actions that must be prevented. 
Refer to any external policies or regulations that state safety issues that affect the product’s design or use. 
Define any safety certifications that must be satisfied.}

Um die Nutzerdaten ausreichend zu schützen, werden hohe Sicherheitsanforderungen benötigt. Dazu gehört zum einen die Verschlüsselung der Kommunikation, aber auch das verschlüsselte ablegen sensibler Daten. Außerdem wird das System regelmäßig auf den neusten Stand gebracht um mögliche aufkommenden Sicherheitslücken durch externe Bibliotheken so schnell wie möglich zu schließen. Mögliche Brute Force Attacken werden eingeschränkt ohne dem User zu viel Freiraum zu nehmen durch die in \ref{acc} gegebenen Funktionalen Anforderungen.\\
Falls in eines unserer Systeme eingedrungen werden sollte, müssen sensible Daten trotzdem so gut wie möglich geschützt sein. Deshalb werden die Kontodaten, Passwörter und Adressen nicht zusammen gespeichert.
Weiterhin werden die Account-, Bezahl- und Verbrauchsdaten jeweils in verschiedenen Datenbanken gespeichert.
%Risikobewertung vs Kosten??

Die Daten werden an mindestens zwei Verschiedenen Orten 
abgespeichert, damit bei einem Verlust von Daten % wie können wir das verhindern?
ein Backup vorhanden ist.
Diese Backups werden alle 5 Minuten erstellt, um den Datenverlust so gering wie möglich zu halten. %sollen wir das begründen?




\subsection{Datenschutzanforderungen}
Die Datenschutzanforderungen sollen der Datenschutzgrundverordnung entsprechen. 
Bei einem Datenbankbreach sollen die sensiblen Daten der Nutzer nicht gefährdet werden. 
Deswegen dürfen die Kontodaten, Passwörter und Adressen nicht zusammen gespeichert werden. 
Weiterhin werden die Account-, Bezahl- und Verbrauchsdaten jeweils in verschiedenen Datenbanken gespeichert. 
Passwörter werden zusätzlich nur als Hashwerte abgespeichert, welche dann nur noch abgeglichen werden müssen.\\
Es sollen möglichst wenig persönliche Daten erhoben werden müssen und viele sollen optional sein:

\begin{enumerate}
    \item Ein Nutzername und das zugehörige Passwort müssen abgespeichert werden um den Nutzer eindeutig zuzuordnen
    \item Der Nutzer kann seine Kreditkartendaten abspeichern, muss er aber nicht
    \item Der Nutzer kann seine Adresse angeben, muss er aber nicht
    \item Zu Immobilien können Adressen gespeichert werden sind aber optional
    \item Eingetragene Daten zu einer Immobilien werden pro Benutzer in einer Datenbank gespeichert
\end{enumerate}
Die Verbräuche eines Nutzers werden zusätzlich anonym in einer Datenbank für Vergleiche festgehalten. Alle Daten von Gas-, Strom- und Wasserverbräuche werden ohne Nutzerbezug abgespeichert. Diese Daten dienen Verbrauchsstatistiken mit denen sich jeder Nutzer vergleichen kann. Es wird Wert darauf gelegt, dass sich diese Daten nicht zurückverfolgen lassen, aufgrund der Regionalität. Sollten zu wenig Daten für einen regionalen Vergleich vorliegen, wird dieser nicht angeboten.

\subsection{Software-Qualitätsattribute}
\textit{Specify any additional quality characteristics for the product that will be important to either the customers or
the developers.
Some to consider are: adaptability, availability, correctness, flexibility, 
interoperability, maintainability, portability, reliability, reusability, 
robustness, testability, and usability. Write these to be specific, quantitative, 
and verifiable when possible.
At the least, clarify the relative preferences for various attributes, 
such as ease of use over ease of learning.}
% Auf 4-5 wichtige Eigenschaften einigen, welche einander nicht ausschließen
% Zur Auswahl: 
% Anpassungsfähigkeit, Verfügbarkeit, Richtigkeit, Flexibilität,
% Kombatibilität, Wartungsfähigkeit, Portabel, Zuverlässig,
% Wiederverwendbarkeit, Robust, Testbar, Nützlich/ Benutzerfreundlich? 

% Spezifisch, messbar, bestätigbar?

\begin{itemize}
    \item \textbf{Verfügbarkeit: }
    Die abzurufenden Daten sollten jederzeit verfügbar sein,
    damit der Benutzer immer die Möglichkeit hat, 
    seine Verbäuche einzusehen und vergleichen zu können.
    Im Normalbetrieb sollte das System innerhalb eines Monats maximal für eine Stunde nicht erreichbar sein.
    
    Um das zu gewährleisten, sollte man jeden Tag an mehreren Uhrzeiten auf das System zugreifen
    und dokumentieren, ob man vom System eine Antwort bekommt und wie lange das dauert. 
    Daraus wird analysiert, ob es bestimmte Uhrzeiten, Wochentage oder Monate gibt, 
    an denen mit einem höheren Betrieb zu rechnen ist, als an den anderen und es kann dementsprechend das
    System auf diese Zeiten angepasst werden, damit die Verfügbarkeit trotzdem gewährleistet werden kann. 
   
    Es sollte auch regelmäßig getestet werden, ob das System im Hochlastbetrieb erreichbar ist, 
    indem man auf dem System eine besonders hohe Menge an Anfragen simuliert und währenddessen probiert,
    ob das System immer noch verfügbar ist. 
    
    Im Fall eines Ausfalls des Systems sollte es nach spätestens sechs Stunden wieder erreichbar sein.

    \item \textbf{Wiederherstellbarkeit: }
    Beim Versagen einer Komponente oder des ganzen Systems soll nach zwei Stunden bemerkt werden, 
    dass ein Problem aufgetreten ist und nach spätestens sechs Stunden muss das System wieder verfügbar sein.
    Weiterhin sollten alle fünf Minuten ein Back-up ((angelegt)) werden, um den Datenverlust während eines Ausfalls
    auf nur 5 Minuten vor Versagen des Systems zu beschränken. 
    ALT: Weiterhin sollen nicht mehr als fünf Minuten an Daten verloren gehen.
    
    Dazu sollten Trainingsszenarios geplant werden, bei denen ein Versagen des Systems simuliert wird. 
    Es wird gemessen, wie lange es dauert, das System wieder verfügbar zu machen, und es kann beobachtet werden,
    wie aktuell die Daten sind, welche vor dem Ausfall gespeichert wurden. 
    
  
    \item \textbf{Wartungsfähigkeit: }
    Die Administratoren und der Support sollten korrekte Verbrauchsdaten 
    pflegen und die Anwendung sollte jederzeit an ein Abo-Modell anpassbar sein.
    ((messbar?))
    
    Das heißt, der Code muss übersichtlich, nachvollziehbar und gut dokumentiert sein. 
    Es sollte ein einheitlicher Standart festgelegt werden, welcher von höheren Instanzen durchgesetzt wird. 
    Um die Wartbarkeit zu testen, soll in einem Trainingsszenario ein Fehler im Code simuliert werden, 
    und es wird beobachtet, wie lange es dauert bis das Problem gefunden und beseitigt wird. 
    

    \item \textbf{Benutzerfreundlichkeit: }
    Die angezeigten Daten sollten möglichst viele Kundenbedürfnisse 
    befriedigen, übersichtlich und leicht verständlich sein.
    Man kann den Nutzer stichprobenartig bitten, eine Bewertung oder Kritik zu hinterlegen zur Benutzerfreundlichkeit. 
    Weiterhin kann analysiert werden, ob bei dem Support Fragen zum Ausführen oder Finden einer konkreten Funktionalität besonders häufig vorkommt.
    Sehr wichtige Funktionen ((wie zb)) sollten sehr intuitiv ausführbar sein und speziellere, weniger interessante Funktionen können zweitrangig 
    betrachtet werden, damit das Design simpel und nicht überfüllt ist.
    ((messbar??))
    
    Um ein geeignetes User Interface festzulegen, sollten Befragungen am Menschen der Zielgruppe stattfinden,
    welche verschiedene Designs bewerten.
    Oder man kann sie darum bitten, in den verschiedenen ((Interface)) eine konkrete Funktion zu finden und auszuführen.
    Es wird die Zeit gemessen, die benötigt wurde um die Funktion auszuführen.
    Das Design, mit dem die Befragten am schnellsten arbeiten konnten, ist für sie das Intuitivste und sollte als finales Design in Betracht gezogen werden. 
    
   
\end{itemize}
