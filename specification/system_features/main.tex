\section{Systemfunktionen}
\label{system_features}

Wir haben die Systemfunktionen nach Nutzergruppen aufgeteilt.

\subsection{Systemfunktionen für alle Nutzergruppen}
\subsubsection{Anmeldung}
\paragraph{Beschreibung und Priorität}

Bevor ein dem System bereits bekannter Nutzer die Funktionen seiner Nutzergruppe ausführen kann,
muss er sich zunächst authentifizieren.
Die Funktion hat eine sehr hohe Priorität, da sonst Nutzer und Nutzergruppen nicht voneinander unterschieden werden können.

\paragraph{Sequenzen von Benutzeraktionen und Systemantworten}

Der Nutzer gibt in einer Anmelde-Maske seinen Benutzernamen oder seine E-Mail-Adresse und sein Passwort ein
und bestätigt dies.
Ist ANM4 erfüllt so ist der Nutzer nun eingeloggt und wird zum Home-Bildschirm weitergeleitet.
War dies nicht der Fall, so wird eine Fehlermeldung mit den möglichen Ursachen angezeigt.

\paragraph{Funktionale Anforderungen}
\begin{itemize}
	\item \textbf{ANM1} Darstellen der Anmelde-Maske.
	\item \textbf{ANM2} Validieren der Eingabe also Überprüfung von ANM4 durch ggf. Abgleich mit der Datenbank.
	\item \textbf{ANM3} Auf nicht erfüllte Anforderungen (ANM4) in der Anmelde-Maske aufmerksam machen,
						falls diese nicht erfüllt sind.
	\item \textbf{ANM4} Der Benutzername oder die E-Mail-Adresse muss im System vorhanden sein und das Passwort
						bzw. dessen Hash müssen übereinstimmen.
\end{itemize}


%----------------------------------------------------------------------------------------------------------------
% 	BENUTZER
%----------------------------------------------------------------------------------------------------------------
\subsection{Benutzer}
\subsubsection{Accounterstellung}\label{acc}
\paragraph{Beschreibung und Priorität}
Diese Funktion ermöglicht es einer Person, die an dem Produkt interessiert ist,
einen Account anzulegen und anzufangen das Produkt zu nutzen.
Die Priorität dieser Funktion ist sehr hoch, da ohne sie das Produkt keine neuen Nutzer gewinnen kann.
\paragraph{Sequenzen von Benutzeraktionen und Systemantworten}
Der Interessierte gibt in der Anmelde-Maske seinen frei gewählten Benutzernamen, seine E-Mail-Adresse
und schließlich ein Passwort ein.
Nach einem Knopfdruck wird, wenn die Eingaben die funktionalen Anforderungen REG4, REG5, REG6 erfüllen,
der Benutzer nun auf die Startseite weitergeleitet.
Wenn nicht, wird der Benutzer über die nicht erfüllten Anforderungen in der Anmelde-Maske informiert
und kann daraufhin seine Eingaben anpassen.
\paragraph{Funktionale Anforderungen}
\begin{itemize}
	\item \textbf{REG1} Darstellen der Anmelde-Maske
	\item \textbf{REG2} Validieren der Eingabe also Überprüfung von REG4, REG5, REG6
						durch ggf. Abgleich mit der Datenbank
	\item \textbf{REG3} Auf nicht erfüllte Anforderungen (REG4, REG5, REG6) in der Anmelde-Maske aufmerksam machen,
						falls diese nicht erfüllt sind.
	\item \textbf{REG4} Keine 2 Benutzer dürfen denselben Benutzernamen haben.
	\item \textbf{REG5} Das Passwort muss dem folgenden Standard genügen:
	      \begin{itemize}
		      \item Mindestlänge: 8 Zeichen
		      \item Enthält min. einen deutschen Großbuchstaben (A-Z, Ä, Ö, Ü)
		      \item Enthält min. einen deutschen Kleinbuchstaben (a-z, ä, ö, ü, ß)
		      \item Enthält min. eine arabische Ziffer (0-9)
			  \item Enthält min. ein Sonderzeichen (!, \$, \#, \%, \ldots)
	      \end{itemize}
	\item \textbf{REG6} Die eingegebene E-Mail-Adresse muss im richtigen Format sein.
	\item \textbf{REG7} Abspeichern der neuen Accountdaten.

\end{itemize}

%----------------------------------------------------------------------------------------------------------------
\subsubsection{Verbräuche eintragen}
\label{sysf:verb_eintragen}
\paragraph{Beschreibung und Priorität}
Die Funktion ermöglicht dem Benutzer seine Verbrauchsdaten in die Datenbank einzutragen,
welche später für weitere Features vonnöten sind.
Neben dem manuellen Eintragen gibt es auch die Möglichkeit der automatischen Erfassung der Daten
mittels eines externen Geräts, welches der Benutzer bei sich Zuhause installiert.
Sobald dies passiert ist, übermittelt das Gerät jeden Tag die aktuellen Daten.

\paragraph{Sequenzen von Benutzeraktionen und Systemantworten}
Beim manuellen Eintragen trägt der Benutzer in einer Webmaske seine einzelnen Verbräuche ein und kann mit einem Klick
auf den Button \gquote{hochladen} seine Daten der Datenbank hinzufügen.
Bevor dies geschieht, wird aber überprüft, ob die Daten neu und sinnig sind.

Die konkreten Anforderungen an die Schnittstelle für das externe Gerät zum automatischen Eintragen von Verbräuchen
sind noch zu diskutieren.

\paragraph{Funktionale Anforderungen}
Noch zu diskutieren. % Dürfen wir das einfach so schreiben??

%--------------------------------------------------------------------------------------------------------------------
\subsubsection{Statistiken einsehen}
\paragraph{Beschreibung und Priorität}
Die Funktion ermöglicht es dem Benutzer seine eingetragenen Verbräuche (siehe~\ref{sysf:verb_eintragen})
zu visualisieren.
Da diese Funktion den größten Nutzen für die Benutzer bringt, hat auch diese Funktion eine hohe Priorität,
sie ist aber nicht essenziell für das Funktionieren des Systems.

\paragraph{Sequenzen von Benutzeraktionen und Systemantworten}

Der Nutzer geht auf den Menüpunkt Statistiken und wählt den gewünschten Verbrauchszeitraum und die Verbrauchsart aus.
Außerdem kann er sich für eine Art der Visualisierung (z.B. Säulendiagramm, Liniendiagramm, oder Tabelle) entscheiden.
Die gewünschten Daten werden daraufhin dargestellt.

\paragraph{Funktionale Anforderungen}
Noch zu diskutieren.


%----------------------------------------------------------------------------------------------------------------

\subsubsection{Verbraucher verwalten}

\paragraph{Beschreibung und Priorität}
Es soll möglich sein, dass neben dem manuellen Eintragen der Verbräuche und automatischen Eintragen
durch ein externes Gerät (siehe~\ref{sysf:verb_eintragen}), die Verbräuche durch einen Verbraucher zu generieren.
% Dieser wird dann einen Verbauch jeden Tag generieren.
Das Verwalten von Verbraucher hat eine eher niedrige Priorität, da die Kernfunktionalität der Verbräche schon
mit manuellen Eintragen abgedeckt ist.
\paragraph{Sequenzen von Benutzeraktionen und Systemantworten}
Wenn der Nutzer einen bestehenden Verbrauch bearbeiten will,
soll dieser aus einer Liste aller erstellten Verbrächer auswählbar sein.
Dabei ist jeder Verbraucher einer Immobilie zugeordnet, was in der Liste auch kommuniziert sein soll.
Es kann natürlich auch ein Verbraucher aus der Liste gelöscht werden.
% Falls ein bestehender Verbraucher bearbeitet werden soll, soll dieser aus einer Liste aller erstellten Verbrächer auswählbar sein. Dabei ist jeder Verbraucher einer Immobilie zugeordnet, was in der Liste auch kommuniziert sein soll. Es kann natürlich auch ein Verbraucher aus der Liste gelöscht werden.

Falls ein Verbraucher bearbeitet oder erstellt werden soll, kann man nun den Namen, den täglichen Verbrauch
(in Bezug auf eine Verbrauchszeitreiehe) und die Immobilie eintragen oder bearbeiten.
Schließlich können die Änderungen gespeichert werden.
\paragraph{Funktionale Anforderungen}
\begin{itemize}
	\item \textbf{VERV1} 	Liste aller Verbraucher anzeigen.
							Zusätzlich signalisieren, zu welcher Immobilie der jeweilige Verbraucher assoziiert ist.
	\item \textbf{VERV2} 	Name, assoziierte Immobilie und täglichen Verbrauch anpassen.
	\item \textbf{VERV3} 	Tägliches Erstellen eines Verbrauchs mit dem gespeicherten täglichen Verbrauch.
\end{itemize}


%--------------------------------------------------------------------------------------------------------------------
\subsubsection{Prognosen einsehen}
\paragraph{Beschreibung und Priorität}
Die Funktion Prognosen einsehen dient dem Nutzer dazu, Prognosen für zukünftige Verbräuche einzusehen.
Die Prognosen werden auf Basis von vorherigen Verbräuchen und eingetragenen Verbrauchern gebildet.
Wenn der Verbrauch eines Nutzers im letzten Jahr ungefähr gleich geblieben ist,
wird dies auch für das nächste Jahr prognostiziert.
\paragraph{Sequenzen von Benutzeraktionen und Systemantworten}
Wenn der Nutzer sich Prognosen für seine Verbräuche anzeigen lassen möchte,
muss er zunächst einen vorhandenen Verbrauchstyp auswählen.
Zusätzlich dazu muss der Nutzer einen bestimmten Zeitraum wählen, für die er die Prognose möchte.
Falls dieser Zeitraum zu weit in der Zukunft liegt, oder der Verbrauchstyp kaum oder gar keine Verbräuche besitzt,
können keine Prognosen ermittelt werden.
Wenn Prognosen ermittelt werden können, werden diese dem Nutzer angezeigt.
Ob diese Prognosen in Form eines Gesamtverbrauchs (z.B. 1600 kWh für nächstes Jahr),
oder in Form einer Verbrauchszeitreihe (Diagramm) angezeigt werden, ist noch offen.
\paragraph{Funktionale Anforderungen}
Noch zu diskutieren

%--------------------------------------------------------------------------------------------------------------------
\subsubsection{Kosten einsehen}
\paragraph{Beschreibung und Priorität}
Die Funktion Kosten einsehen ermöglicht dem Nutzer die Kosten zu seinen Verbräuchen einzusehen.
Bevor dies geschehen kann, muss der Benutzer einen Preis pro angegebener Einheit angeben.
Zudem kann er sich bei eigenen Verbrauchstypen auch eigene Kosten eintragen.
Beispiel: Der Nutzer hat den Verbrauchstyp Anzahl an Kaffee pro Tag.
Dazu kann der Nutzer nun einen Preis eintragen: Ein Kaffee kostet einen Euro.
\paragraph{Sequenzen von Benutzeraktionen und Systemantworten}
Der Nutzer muss einen Verbrauchstyp auswählen und dann zu diesem Typ einen Zeitraum und Kosten pro Einheit wählen.
Danach werden ihm die Kosten für den Verbrauch über den Zeitraum angezeigt.
Wenn der Nutzer seinen Verbrauch vergleicht, ist neben dem Vergleich des Verbrauchs auch ein Vergleich der Kosten
sichtbar.
\paragraph{Funktionale Anforderungen}
Noch zu diskutieren

%--------------------------------------------------------------------------------------------------------------------
\subsubsection{Immobilien bearbeiten}
\paragraph{Beschreibung und Priorität}
Der Benutzer kann Immobilien hinzufügen,
bearbeiten und entfernen. %was soll das noch können?
% Ein Benutzer sollte(?) für das Eintragen von Wasser-/Gas-/ Stromverbrauch 
% mindestens eine Immobilie angeben mit Informationen über den Flächeninhalt 
% und Anzahl der Haushaltsmitglieder für eine möglichst genaue Analyse über den Verbrauch des Nutzers verglichen mit dem Durchschnitt. 
Diese Funktion hat eine mittlere Priorität da Verbräuche an eine Immobilie gebunden sind.

\paragraph{Sequenzen von Benutzeraktionen und Systemantworten}
Der Benutzer wählt in seinem Profil aus,
dass er seine Immobilien bearbeiten möchte.
Dabei bekommt er die Option, eine neue Immobilie anzulegen,
oder von vorhandenen Immobilien eine zu bearbeiten.
Wenn er das Erste aussucht, kann man eine Bezeichnung,
Flächeninhalt und Haushaltsgröße % noch was?
eintragen. Die Bezeichnung ist zwingend notwendig, %(?)
die anderen Angaben nicht.
Der Nutzer kann sich dazu entscheiden, den Vorgang abzubrechen.
Wenn der Nutzer eine vorhandene Immobilie bearbeiten will,
dann werden ihm ebenfalls angeboten, Bezeichnung, Flächeninhalt und
Haushaltsgröße zu ändern.
Zusätzlich hat er die Option, die ganze Immobilie zu löschen.
Wenn der Nutzer etwas bearbeitet hat, muss er seine Änderungen bestätigen.
Sobald er das getan hat, ist der Bearbeitungsvorgang abgeschlossen.

Wenn der Nutzer sich für das Löschen entscheidet, wird er final gewarnt,
dass das Löschen einer Immobilie alle damit verbundenen Daten löscht und der Vorgang nicht widerrufbar ist.
%klingt momentan sehr dumm muss geändert werden
Wenn der Nutzer zugestimmt hat, dass ihm das bewusst ist und trotzdem fortfahren will,
dann wird die Immobilie und alle Daten, die damit verbunden waren, aus der Datenbank entfernt.

\paragraph{Funktionale Anforderungen}
Noch zu diskutieren.

%--------------------------------------------------------------------------------------------------------------------
\subsubsection{Abomodell ändern}
\includegraphics[scale=0.5]{activity_abo}

\paragraph{Beschreibung und Priorität}
Der Benutzer kann hiermit zwischen Abomodellen wechseln.
Geplant sind die Abomodelle FREE, STANDARD und PROFESSIONAL,
wovon Ersteres kostenlos ist und standardmäßig ausgewählt sein sollte.
Diese Funktion enthält neben dem Auswählen des Modells die Zahlung und Überprüfung vertraglicher Kündigungsfristen.
Die Funktion gehört nicht zu den Kernanforderungen jedoch kann ohne sie kein Umsatz erzielt werden
und hat daher trotzdem eine hohe Priorität.
\paragraph{Sequenzen von Benutzeraktionen und Systemantworten}
Nach der Auswahl des gewünschten Abomodells muss man, falls sich der Nutzer für ein kostenpflichtiges Modell entscheidet,
einen Startzeitpunkt (nach Richtlinie ABO3) entscheiden.
Nach der Eingabe von Rechnungsadresse, die, falls sie für den Benutzer schon bekannt ist,
auch schon voreingetragen sein sollte, kann der Benutzer eine präferierte Zahlungsmethode anwählen,
wodurch er zu einem externen Zahlungsdienstleister weitergeleitet wird.
Schließlich kann der Prozess abgeschlossen werden und das neue Abomodell wird zum angegebenen Startzeitpunkt gesetzt.
Eine Zahlungsbestätigung wird versendet.
Der Vorgang kann zu jedem Zeitpunkt abgebrochen werden.

\paragraph{Funktionale Anforderungen}
\begin{itemize}
	\item \textbf{ABO1} Darstellen der Maske zum Auswählen des Abomodells und Eintragen von Startzeitpunkt
						und Rechnungsdresse.
	\item \textbf{ABO2} Validieren der Eingabe des Startzeitpunkts (nach ABO3) und Reflektieren möglicher
						Fehler in der Maske.
	\item \textbf{ABO3} Der Startzeitpunkt muss folgenden Richtlinien genügen:
	      \begin{itemize}
		      \item Er muss in der Zukunft liegen oder am derzeitigen Tag sein
		      \item Falls der Kündigungszeitraum eines laufendes Abos noch nicht abgelaufen ist,
			  		muss der Startzeitpunkt nach Ablauf dieses Zeitraums liegen
	      \end{itemize}
	\item \textbf{ABO4} Grundlegende Format-Überprüfungen der Rechnungsadresse
						(z.B. Postleitzahl besteht aus Zahlen etc.).
						Nichteinhaltung sollen in der Maske kommuniziert werden.
	\item \textbf{ABO5} Eine Verbindung zu den externen Zahlungsanbietern muss aufgebaut werden.
	\item \textbf{ABO6} Schaffen einer Möglichkeit in jedem Schritt den Prozess abzubrechen.
	\item \textbf{ABO7} Versenden einer Zahlungsbestätigung.
	\item \textbf{ABO8} Setzen des neuen Abomodells in den Benutzerdaten zum angegebenen Startzeitpunkt.
\end{itemize}


%--------------------------------------------------------------------------------------------------------------------
\subsubsection{Account für Support freischalten}
\label{sys_feat:freischalten}
\paragraph{Beschreibung und Priorität}
Diese Funktion ermöglicht es dem Benutzer bei Problemen seinen Account für den Support freizuschalten,
sodass der Support Änderungen vornehmen kann.
Dies ermöglicht einen Support, der zugleich datenschutz- und nutzerfreundlich ist.
\paragraph{Sequenzen von Benutzeraktionen und Systemantworten}
Der Benutzer hat mit der Nutzung des Systems ein Problem und benötigt Hilfe.
Nachdem er die FAQ zur Handhabung des Systems gelesen hat und er dort keine Lösung für sein Problem finden konnte,
nutzt er die Supportfunktion, wo er mit einem Mitarbeiter in Kontakt kommt.
Zuerst versucht der Support dem Nutzer ohne weitere Daten weiterzuhelfen.
Sollte dies das Problem immer noch nicht lösen, muss der Nutzer seinen Account für den Support freischalten.
Dem Nutzer wird ein Code generiert, den er dem Support durchgibt und den der Support benötigt,
um sich berechtigten Zugang zum Nutzeraccount zu verschaffen.
Nun kann der Support dem Nutzer optimal weiterhelfen, indem er Zugriff auf seinen Account hat.
\paragraph{Funktionale Anforderungen}
\begin{itemize}
	\item \textbf{REG1} Laufende Internet- und/oder Telefonverbindung.
	\item \textbf{REG2} Der generierte Code für den Nutzer besteht aus Zahlen und Buchstaben.
	\item \textbf{REG3} Der generierte Code von Nutzer muss unbedingt mit dem benötigten Code des Supports übereinstimmen.

\end{itemize}


%--------------------------------------------------------------------------------------------------------------------
\subsubsection{Account verwalten}
\paragraph{Beschreibung und Priorität}
Das Feature erlaubt es dem Nutzer seine Accountdaten einzusehen, zu ändern und gegebenenfalls zu löschen
und es hat eine sehr hohe Priorität.

\paragraph{Sequenzen von Benutzeraktionen und Systemantworten}
Der Nutzer wählt den Menüpunkt \gquote{Account Verwalten} aus.
Dort sind alle Optionen AVW1 - AVW6 einsehbar und änderbar.

\paragraph{Funktionale Anforderungen}
\begin{itemize}
	\item \textbf{AVW1} Name.
	\item \textbf{AVW2} Adresse.
	\item \textbf{AVW3} Zahlungsmethode.
	\item \textbf{AVW4} Nutzername.
	\item \textbf{AVW5} Email.
	\item \textbf{AVW6} Passwort.
						Hier sollte das aktuelle Passwort jedoch nicht einsehbar sein
						(da es nicht gespeichert werden soll).
\end{itemize}



%--------------------------------------------------------------------------------------------------------------------

\subsubsection{Verbrauch vergleichen}
\includegraphics[scale=0.5]{activity_vergleich}
% Bei Abfrage einer Vergleichszeitreihe auch darauf eingehen, dass diese auch von extern gekommen sein könnte
\paragraph{Beschreibung und Priorität}
Die Funktion Verbrauch vergleichen ermöglicht dem Nutzer seinen Verbrauch mit seiner eigenen Verbrauchshistorie,
oder mit den Verbräuchen anderer zu vergleichen.
Falls man sich mit den Verbräuchen von anderen vergleichen möchte, kann man Vergleichstags auswählen
(siehe~\ref{sec:vergl_tags}).
Dann wird eine Vergleichszeitreihe neben der eigenen Verbrauchszeitreihe visualisiert.
\paragraph{Sequenzen von Benutzeraktionen und Systemantworten}
Zunächst muss der Nutzer einen Verbrauchstyp auswählen.
Danach wählt der Nutzer aus, ob er sich mit seinen eigenen Verbräuchen, oder mit den Verbräuchen anderer vergleichen möchte.
Wenn er ersteres auswählt, sucht sich der Nutzer zwei Zeiträume aus, die er gegeneinander vergleichen möchte.
% Möchte er sich nur mit sich selber vergleichen, wählt der Nutzer diese Funktion und zwei Zeiträume aus,
% die er gegeneinander vergleichen möchte.
Daraus werden zwei Zeitreihen generiert, die dann in einer Statistik visualisiert werden.
Möchte der Nutzer sich mit Anderen vergleichen, wählt er einen oder mehrere Vergleichstags (siehe~\ref{sec:vergl_tags}) aus.
Beispiel: Der Nutzer möchte sich mit anderen Haushalten aus seiner Region vergleichen.
Der Nutzer kann den Namen seiner Region und die Wohnfläche des Hauses angeben.
Auch hier muss der Nutzer einen Zeitraum wählen, in dem der Nutzer sich vergleichen will.
Dann wird eine Vergleichszeitreihe mit der Vergleichsgruppe in dem gegebenen Zeitraum erstellt und visualisiert.
\paragraph{Funktionale Anforderungen}
Noch zu diskutieren


%--------------------------------------------------------------------------------------------------------------------

\subsubsection{Support Ticket erstellen und interagieren}
\label{sysf:ticket_erstellen}
% Hier also auch Support-Ticket Schließen
\paragraph{Beschreibung und Priorität}
Dem Benutzer wird die Möglichkeit gegeben, Support Tickets zu erstellen und somit Kontakt zu einem Supporter aufzunehmen
der ihm gegebenenfalls bei einem Problem bezüglich der Software helfen kann.
Wenn ein Problem gelöst wurde, kann das Support-Ticket auch geschlossen werden.
Diese Funktion ist relevant um die Benutzerfreundlichkeit weiter zu erhöhen.
\paragraph{Sequenzen von Benutzeraktionen und Systemantworten}
Der Benutzer wählt den Knopf \gquote{Support kontaktieren} um mit dem Support in Kontakt zu treten.
Hier hat er schon die Möglichkeit, sein Problem konkreter zu schildern.
Nach anschließender Bestätigung durch einen weiteren Knopfdruck wird das Ticket erstellt
und der Benutzer erhält eine Benachrichtigung, dass sich zeitnah jemand um seine Angelegenheit kümmern wird.
Sobald ein Supporter Zeit hat, sich mit dem Problem zu beschäftigen, kann der Supporter das Ticket annehmen (siehe~\ref{sysf:ticket_annehmen})
und ein Live-Chat zwischen beiden wird eingerichtet.
Wenn sich das Problem erledigt hat, bekommt der Support die Möglichkeit den Live-Chat zu schließen.
Es soll auch einen Knopf geben, um das Support-Ticket zu schließen.
\paragraph{Funktionale Anforderungen}
\begin{itemize}
\item \textbf{TICK1} Bereitstellung eines Interfaces um das Ticket zu erstellen.
\item \textbf{TICK2} Möglichkeit für den Benutzer den Live-Chat zu schließen.
\end{itemize}

%--------------------------------------------------------------------------------------------------------------------
%		SUPPORT
%--------------------------------------------------------------------------------------------------------------------

\subsection{Support}
\subsubsection{Support-Ticket eines Benutzers annehmen und \label{sysf:ticket_annehmen}
interagieren}
% Hier kann einfach nach akzeptieren gesagt werden siehe 4.2.11 und auch ein verweis auf 4.3.2 wäre gut.
\paragraph{Beschreibung und Priorität}
Es ist wichtig, dass der Support nicht unkontrolliert auf alle Accounts der Nutzer zugreifen kann.
Daher müssen vor einer Interaktion mit den Daten der Nutzer immer ein Ticket vom Benutzer erstellt werden (siehe~\ref{sysf:ticket_erstellen}).
Nachdem das gelungen ist und das Ticket vom Support angenommen wurde, hat dieser die Berechtigung
den Account zu verwalten (siehe ~\ref{sysf:support_account_verwalten}) und dem Benutzer optimal zu helfen.
\paragraph{Sequenzen von Benutzeraktionen und Systemantworten}
Nachdem der Benutzer ein Support-Ticket geöffnet hat (siehe ~\ref{sysf:ticket_erstellen}), muss der Support
nur noch das Ticket annehmen, durch das Klicken auf einen Button.
Danach öffnet sich ein Live-Chat zwischen beiden und der Support hat Zugriff auf die Daten des Benutzers.
Im Fall, dass es nützlich sein sollte, kann der Supporter das Bearbeiten des Benutzer-Accounts anfragen (siehe~\ref{sysf:support_account_verwalten}).
Nachdem das Problem gelöst wurde, kann der Supporter den Live-Chat und somit auch das Ticket schließen,
durch das Klicken auf einen weiteren Button
und der Support verliert den Zugriff auf die Daten des Nutzers.
% TODO: und der Support verliert wieder den Zugriff auf den Nutzer
\paragraph{Funktionale Anforderungen}
\begin{itemize}
	\item \textbf{SUPA1} Möglichkeit für den Supporter den Live-Chat zu schließen. 
	\item \textbf{SUPA2} Möglichkeit für den Supporter das Support Ticket zu schließen.
\end{itemize}

%--------------------------------------------------------------------------------------------------------------------
\subsubsection{Account eines Nutzers verwalten}
\label{sysf:support_account_verwalten}
\paragraph{Beschreibung und Priorität}
Ein Support muss in der Lage sein, etwaige Benutzerdaten auf Anfrage des jeweiligen Benutzers zu ändern.
Dafür muss allerdings das Einsehen der Benutzerdaten für den Support freigeschaltet werden (siehe~\ref{sys_feat:freischalten}).
Diese Funktion ist nicht teil der Kernanforderungen, jedoch muss es unbedingt Teil der ersten Version sein,
da die Kundenzufriedenheit davon abhängig ist.
Daher ergibt sich eine mittlere Priorität.
\paragraph{Sequenzen von Benutzeraktionen und Systemantworten}
Der Support kann den Benutzer auswählen, dessen Stammdaten bearbeitet werden sollen.
Dabei werden dem Support nur diejenigen Benutzerdaten angezeigt, die die Bearbeitung freigeschaltet haben.
Danach wird eine Oberfläche angezeigt, die der Oberfläche zum Bearbeiten der eigenen Stammdaten eines Benutzers
ähnelt.
Hier können nun die Änderungen vorgenommen und gespeichert werden.
\paragraph{Funktionale Anforderungen}
\begin{itemize}
	\item \textbf{SMG1} Zeige alle Benutzer, die die Bearbeitung nach ~\ref{sys_feat:freischalten} freigeschaltet haben.
	\item \textbf{SMG2} Bearbeiten aller Stammdaten eines Benutzers in der Oberfläche ermöglichen.
	\item \textbf{SMG3} Speichern der veränderten Daten.
\end{itemize}



%--------------------------------------------------------------------------------------------------------------------
%		ADMIN
%--------------------------------------------------------------------------------------------------------------------


\subsection{Admin}
\subsubsection{Abomodelle modifizieren}
\paragraph{Beschreibung und Priorität}
Der Admin kann hiermit alle drei verfügbaren Abomodelle (FREE, STANDARD und PROFESSIONAL) grundlegend ändern.
Alle Abomodelle haben verschiedene Funktionen, welche der Admin anpassen kann.
Der Admin hat die Möglichkeit, die Kosten der verschieden Modelle anzupassen, Funktionen freischalten
und Funktionen einzuschränken oder sogar ganz zu entfernen.
Falls der Admin die Kosten eines Modells ändert, oder Funktionen hinzufügt oder entfernt,
müssen gesetzliche Regelungen für Vertragsänderungen eingehalten werden.
Das heißt, Nutzer dieses Abomodells müssen benachrichtigt werden und es muss ihnen eine Möglichkeit gegeben werden,
das Abomodell zu kündigen.
Das passiert unabhängig davon, ob der Admin die Kosten erhöht oder verringert, oder Funktionen hinzufügt oder entfernt.

\paragraph{Sequenzen voinaktionn Admen und Systemantworten}
Falls der Admin Abomodelle modifizieren will, muss der Admin gemäß ABOM3 einen Startzeitpunkt wählen,
ab dem die Änderungen gültig werden.
Dem Admin soll bekannt gemacht werden, ob der Zeitpunkt zu kurzfristig gewählt ist.
Danach kann der Admin aus folgenden Funktionen wählen:
\begin{itemize}
	\item Vorhandene Funktion bearbeiten oder entfernen (Funktion muss in dem Abomodell momentan verfügbar sein).
	\item Neue Funktion hinzufügen.
	\item Den Preis des Abomodells ändern.
\end{itemize}
Wenn der Admin sich für eine der Funktionen entschieden hat, kann er diese ausführen und bestätigen,
sofern alles den Anforderungen entspricht.
Der gesamte Vorgang kann jederzeit auch abgebrochen werden.
Fährt der Admin fort, speichert das System die Änderungen und wendet diese zu dem gegebenen Zeitpunkt an.

\paragraph{Funktionale Anforderungen}
\begin{itemize}
	\item \textbf{ABOM1} Darstellen der Maske zum Auswählen des Abomodells und Eintragen von Startzeitpunkt.
	\item \textbf{ABOM2} Funktion entsprechend dem gewählten Abomodell muss änderbar und löschbar sein.
	\item \textbf{ABOM3} Funktion soll hinzufügbar sein.
	\item \textbf{ABOM4} Ein neuer Preis für das Abomodell soll wählbar sein.
	\item \textbf{ABOM5} Der Startzeitpunkt muss in der Zukunft liegen und muss den gesetzlichen Regelungen entsprechen.
	\item \textbf{ABOM6} Validieren der Eingabe des Startzeitpunkts (nach ABO5) und Reflektieren möglicher Fehler
							in der Maske.
\end{itemize}

%--------------------------------------------------------------------------------------------------------------------

\subsubsection{Support- und Adminaccount erstellen}
\paragraph{Beschreibung und Priorität}
Ein Admin soll in der Lage sein Support- und Adminaccounts zu erstellen.
Der Supportaccount kann von dem Support-Team verwendet werden um Zugang zu dem System zu erhalten.
Über diesen Account kann der Support die Systemfunktionen des Supports nutzen.
Ein Adminaccount hätte dieselben Rechte, wie der erstellende Adminbenutzer.
\paragraph{Sequenzen von Adminaktionen und Systemantworten}
Um einen Support- oder Adminaccount zu erstellen soll eine Maske erstellt werden.
In der Maske gibt es die Felder: Name, Passwort und E-Mail sowie die Wahl, ob es sich um ein Support- oder Adminaccount
handelt.
Wenn der Admin alle Felder ausgefüllt hat, kann er auf der Maske einen Bestätigungsknopf drücken
und der Account wird dadurch angelegt.
Zusätzlich soll es auf der Maske einen \gquote{Schließen} Knopf geben, falls der Admin die Aktion abbrechen will.
\paragraph{Funktionale Anforderungen}
Noch zu diskutieren.

% TODO: Accounts wieder löschen?

%--------------------------------------------------------------------------------------------------------------------

\subsubsection{Vergleichstags festlegen}\label{sec:vergl_tags}
\paragraph{Beschreibung und Priorität}
Der Admin soll in der Lage sein, Vergleichstags festzulegen.
Beispiele dafür wären regionale Vergleiche, Vergleiche in Haushaltsgröße, Vergleiche mit ähnlicher Immobiliengröße,
Vergleiche mit Immobilienalter und Vergleiche mit dem Alter des Nutzers. 
Später können dann Vergleichszeitreihen (siehe~\ref{sec:vergl_zeitr}) mit dem neuen Tag erstellt werden.
\paragraph{Sequenzen von Adminaktionen und Systemantworten}

Der Admin wählt in einer Maske einen neuen Vergleichstag, den er hinzufügen möchte.
Es muss gewährleistet sein, dass nicht schon ein Vergleichstag unter demselben Namen existiert.
Die Maske kann der Admin schlussendlich bestätigen oder abbrechen.
Bei Bestätigung wird der Vergleichstag erstellt.
\paragraph{Funktionale Anforderungen}
Noch zu diskutieren.

%--------------------------------------------------------------------------------------------------------------------

\subsubsection{Vergleichszeitreihe erstellen}\label{sec:vergl_zeitr}
\paragraph{Beschreibung und Priorität}
Der Admin kann Vergleichszeitreihen anlegen.
Dafür lässt er sich die Daten aller Nutzer zu interessanten und relevanten
Kennzahlen berechnen, welche bei ihm gespeichert werden, 
damit er, wenn ein Nutzer sich vergleichen möchte mit anderen Nutzern,
nur noch die Daten vom Admin geladen werden, statt jedes Mal auf die gesamte Datenbank zuzugreifen.
Alternativ kann der Admin auch aus externen Ressourcen Vergleichswerte eintragen.

\paragraph{Sequenzen von Adminaktionen und Systemantworten}
Der Admin klickt im Menü auf die Schaltfläche \gquote{Vergleichszeitreihe erstellen}.
Dann sieht man auf der Webseite die Möglichkeit \gquote{aus eigenen Nutzerdaten erstellen} oder
\gquote{manuell eintragen}.
Wenn man das Erste auswählt, werden die Daten der Nutzer ausgewertet und mit den zugehörigen Tags verknüpft.
Das kann einige Sekunden dauern also gibt es während der Ladezeit eine Ladegrafik, 
welche dem Admin mitteilt, dass die Seite mit Rechnen beschäftigt ist.
Der Admin kann dann in allen Tags die berechneten Werte einsehen. 
Wenn er auf \gquote{Vergleichszeitreihe aktualisieren} klickt,
wird er danach gefragt, ob er sich sicher ist, dass er die Vergleichszeitreihe aktualisieren möchte.
Wenn der Admin die Anfrage bestätigt, werden diese Werte als neue Vergleichswerte definiert.
Das heißt, wenn der Nutzer seinen Verbrauch mit dem des Durchschnitts vergleichen möchte,
werden ab sofort diese neu ermittelten Werte verwendet.

Wenn der Admin sich dafür entscheidet, manuell eine Verbrauchszeitreihe anzulegen,
bekommt er die Möglichkeit Tags einzutragen, denen der Eintrag zugeordnet werden soll.
Außerdem kann er die Werte eintragen und Einheiten auswählen, die der Eintrag haben soll.
Wenn er dann auf \gquote{bestätigen} klickt, werden die neuen Werte als
neue Vergleichswerte definiert in den angegebenen Tags.
\paragraph{Funktionale Anforderungen}
Es muss gewährleistet werden, dass beim manuellen Eintragen die Tags tatsächlich existieren.
Die eingetragenen Werte werden auf Plausibilität untersucht.

